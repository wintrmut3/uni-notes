\documentclass[twocolumn,10pt]{article}
\usepackage{amsmath}
\usepackage{amsfonts}
\begin{document}
  \section{Subspaces II}
  20 November 2019
  \subsection{Finding Bases}
  With any mxn matrix A, we can row reduce it.
  $$A_{(mxn)} \to R_{REF, (mxn)}$$ And we find that they have certain relationships and similarities. These are because the elementary operations preserve some things. Eg:
  \begin{itemize}
    \item The row spaces are equal: $row(A) = row(R)$. Recall, the \textbf{row space} is equal to the span of the rows of A. If $$A = \begin{bmatrix}1 &2 &-1& 3 \\ 2 & 1 & 1 & 0 \end{bmatrix}$$
    $$row(A) = span \begin{bmatrix}1 \\ 2 \\ -1 \\3 \end{bmatrix}, \begin{bmatrix}2 \\ 1 \\ 1 \\ 0 \end{bmatrix}$$
  \end{itemize}

  When we have some matrix in row echelon form, it becomes very easy to check linear independence. $$A = \begin{bmatrix}1 &2 &-1& 3 \\ 0 & 0 & 1 & 2 \\ 0 & 0 & 0 & 1 \end{bmatrix}$$
    $$row(A) = span \begin{bmatrix}1 \\ 2 \\ -1 \\ 3
    \end{bmatrix}, \begin{bmatrix} 0\\ 0 \\ 1 \\ 2 \end{bmatrix}, \begin{bmatrix} 0\\ 0 \\ 0 \\1 \end{bmatrix}$$

We can put this into an equation, giving each row a parameter: $$a\begin{bmatrix}1 \\ 2 \\ -1 \\ 3
\end{bmatrix} + b\begin{bmatrix} 0\\ 0 \\ 1 \\ 2 \end{bmatrix} + c \begin{bmatrix} 0\\ 0 \\ 0 \\1 \end{bmatrix} = \begin{bmatrix} 0\\ 0\\0\\0\end{bmatrix}$$ Thus,
$$\begin{bmatrix}a \\ 2a \\ -a + b \\ 2b + c \end{bmatrix} = \begin{bmatrix} 0 \\ 0 \\ 0 \\0 \end{bmatrix}$$ We can take a = 0 from the first line, and calculate for the rest of the rows we know that all the coefficients a,b, and c must be zero. Thus, $a=b=c=0$ is the only solution, so it is linearly independent.\\\\ This tells us that \textbf{For any row echelon R, the set of \textit{NONZERO} rows is linearly independent.}\\\\EX.\\
If we want to find a basis for $$span of \begin{bmatrix}1 \\ 2 \\ -1 \\ 1
\end{bmatrix}, \begin{bmatrix} 2\\ 4 \\ -2 \\ 2 \end{bmatrix}, \begin{bmatrix} 1\\ 3 \\ -1 \\0 \end{bmatrix}$$
We can write them as row vectors and row reduce it.
$$\begin{bmatrix}1 & 2 & -1 & 1 \\ 2 & 4 & -2 & 2 \\ 1 & 3 & -1 & 0 \end{bmatrix}\text{is equivalent to} \begin{bmatrix}1 & 2 & -1 & 1 \\ 0 & 1 & 0 & -1 \\ 0 & 0 & 0 & 0 \end{bmatrix}$$
This provides the basis $\begin{bmatrix}1 & 2 & -1 & 1\end{bmatrix}^T, \begin{bmatrix}0 & 1 & 0 & -1 \end{bmatrix}^T$. However, these are not in the original spanning set. \\\\To find vectors that are a bit closer to the original matrix.\\
$$\begin{bmatrix}1 & 2 & 1 \\ 2 & 4 & 3 \\-1 & -2 & -1 \\ 1 & 2 & 0 \end{bmatrix} \text{equiv to}\begin{bmatrix}1 & 2 & 1 \\ 0 & 0 & 1 \\0 & 0 & 0 \\ 0 & 0 & 0 \end{bmatrix} $$ The leading ones designate the columns (R1, R3) of A from which we take the basis, and provide us a basis \textbf{composed} of a subset of the originals. Thus, the basis is $$\begin{bmatrix}1 \\ 2 \\ -1 \\ 1\end{bmatrix} \begin{bmatrix}1 \\ 3 \\-1 \\0 \end{bmatrix}$$.\\
Say we had a 2x4 matrix A. If we multiply $A\vec{x} = \vec{y}$, we know that $\vec{x}$ has to have 4 elements, and that $\vec{y}$ has 2 elements. ($\vec{x}\in\mathbb{R}^2$, $\vec{y} \in\mathbb{R}^4$). \\\\ \\We can conclude that $dim(row(A)) = 0,1,2$ but definitely not 3, so our row space can have at most 2 vectors. \\ Also, the dimension of the column space is equal to the dimension of the row space, and this is the value we call the rank. $$dim(row A) = dim(col A)$$ \\\\
We also know that $im(A) = col(A)$. This is because the span of the columns of in A: \\\\ $col(A) = span\{\vec{c_1}, \vec{c_2}...\vec{c_n}\}$ \\ Also $$im(A) = \{ A\vec{x} | \vec{x} \in \mathbb{R}^n\}$$ But
$$A\vec {x} = [\vec{c_1}, \vec{c_2}...\vec{c_n}]\begin{bmatrix}x_1 \\ x_2 \\ ...\\x_n \end{bmatrix}= \vec{c_1}x_1 ... \vec {c_n}x_n$$ which is equal to our column space! Just interpret $x_i$ as some arbitrary constant. \\\\

\section{Summary}
\subsection{Subspaces}
For any matrix A:
\begin{itemize}
  \item We have 4 names for 3 subspaces (im(A) = col(A))
  \item row(A) = span (rows)
  \item col(A) = span(cols) = im(A)
  \item null(A) = solutions to $A\vec x = 0$
\end{itemize}
\subsection{Connections}
\begin{itemize}
  \item dim(row(A) = dim(col(A))). In the REF form, the number of vectors is the rank, or the number of leading ones.  These are in the same dimension, though in different spaces.
  \item For every column, we either have a leading one or a non-leading variable. The nonleading variables are the parameters, which give us solutions to the null space. Thus $dim(null A) = n - rank(A) = \text{number of columns} - \text{number of leading variables}$. This is known as the \textbf{Rank + Nullity Theorem} which tell us that $$rank(A) + dim(null(A)) = n$$ or: (number of cols with leading variables) + (number of cols with nonleading variables) = (total number of columns)
\end{itemize}
Thus we can conclude some things about a matrix without knowing anything except for the size.
So, for a 2x4 matrix of A: \begin{itemize}
  \item dim(row(A)) = 0, 1, or 2
  \item dim(col(A)) = 0, 1, or 2
  \item dim(null(A)) = 4, 3, or 2 from the previous two points
\end{itemize}

  %Temp matrix $$ \begin{bmatrix} 0 \\ 0 \\ 0 \end{bmatrix} $$
  %Temp matrix $$ \begin{bmatrix} 0 & 0 & 0\\ 0 & 0 & 0 \\ 0 & 0 & 0 \end{bmatrix} $$
  %Temp matrix $$ \begin{bmatrix} 0 & 0 \\ 0  & 0 \end{bmatrix} $$

\end{document}
